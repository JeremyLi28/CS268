\documentclass[11pt]{article}
\usepackage{geometry}
 \geometry{
 a4paper,
 total={210mm,297mm},
 left=20mm,
 right=20mm,
 top=10mm,
 bottom=10mm,
 }
\usepackage{esvect}
\geometry{a4paper} % or letter or a5paper or ... etc
% \geometry{landscape} % rotated page geometry
% See the ``Article customise'' template for come common customisations
\title{CS 268 Intro to Optimization Homework 3}
\author{Chen Li}
%%% BEGIN DOCUMENT
\begin{document}
\maketitle
\section{Problem1} 
I use the Metropolis approach for Simulated Annealing, and set the initial temperature $T = 50$ and its decay ratio equal to 0.98. When $T<0.1$ the loop ends.
\subsection{1(a)}
I choose the graph-2-coloring problem to solve. Firstly I test my optimizer on a simple 5 nodes graph to validate its correctness. Then I use the different size of check board to test the optimizers scalability, Every size I loop 10 times to get the mean of error and error bar. The results are as follows, we can see that the error begins at $4\times 4$ check board, and grow up with the size. 
\begin{center} 
\begin{tabular}{l*{6}{c}r} T & $Size$ &  $error_{estimated}$
 \\ \hline $50$ & 4 & $0.0\pm 0.0$
\\ $$ & 9 & $0.0\pm 0.0$
\\ $$ & 16 & $0.80\pm 0.55$
\\ $$ & 25 & $2.80\pm 0.95$
\\ $$ & 36 & $6.60\pm 1.19$
\\ $$ & 49 & $16.30\pm 1.59$
\\$$ & 64 & $20.50\pm 1.58$
\\ $$ & 81 & $34.60\pm 2.13$
\\ $$ & 100 & $42.70\pm 2.09$
\\ \end{tabular}
 \end{center}
\subsection{1(b)}
I generalize the graph-2-coloring problem to graph-N-coloring problem by simply change the flip approach from flip between 0 and 1 to flip between 0 and N. And I test it on the simple graph to validate is correctness.
\section{Problem2}
I use Metropolis approach for the float version. I choose the initial step size to be 0.5, when the accept rate  $q$ is great than 0.5 or less than 0.3, adjust the ratio accordingly. I set the const factor of step size adjustment to be 1.2. Also, when $q$ is not in the $0.3-0.5$ range, the move would not be accepted (this approach seems increase the performance). Also, every time I randomly choose one dimension to flip( apply step). The result shows that the coordinate descent method is more accurate in this situation but it cost more time. I test it on the function $f(x,y) = (x-1)^2 + (y-1)^2$.
\begin{center} 
\begin{tabular}{l*{6}{c}r} Function &  $Error$ & $Time$
 \\ \hline $SA$ &  $0.049$ & $0.0198$
\\ $Coordinate Descent$ & $1.241e^{-6}$ & $0.0211$ 
\\ \end{tabular}
 \end{center}
\section{Implementation}
In $hw3.py$,  the $SA()$ implement the Simulated Annealing Method for 2 or more dimension. $SAReal()$ implement the Simulated Annealing Method for real value continues function. The results can be find in the $test\_results$ folder or print on the screen.


















\end{document}